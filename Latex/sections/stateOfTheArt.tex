%!TEX root = ../main.tex
\section{State Of The Art}

\subsection{Spherical SLAM}

While there exist camera-based approaches~\cite{9233654}, to the best of our knowledge, SLAM with spherical robots and laser scanners was not done before.
However, laser-based SLAM algorithms for motions in six degrees of freedom (DoF) have been thoroughly studied.
For outdoor environments~\cite{JFR2006} provides a first baseline.
Adding a heuristic for closed-loop detection and a global relaxation Borrmann et al. yield highly precise maps of the scanned environment~\cite{RAS2007}.
Thereby they reduce the position and orientation error of the scanning poses by orders of magnitude in the range of centimeters and hundreds of a degree.
Zhang et al. propose a real-time solution to the SLAM problem in~\cite{Zhang2014}.
They achieve the performance at a lower computational load by dividing the SLAM algorithm into two different algorithms: one performing odometry at a high frequency but low fidelity and another running at a lower frequency performing fine matching and registration of the point clouds.
More recently Dröschel et al. also propose an online method using a novel combination of a hierarchical graph structure with local multi-resolution maps to overcome problems due to sparse scans~\cite{Droeschel2018}.
Another intriguing example is the NavVis VLX system~\cite{navvis}.
They perform real time SLAM on a mobile, wearable platform, producing colored point clouds with high accuracy by combining it with a camera system.
However, this platform has to be moved by a human operator. 
In particular, they are able to achieve one sigma of measurements within three millimeters difference the measurements of a ground truth terrestrial laser scan~\cite{navvisAcc}. 
They also employ artificial markers that can be placed in the environment, which get recognized by the camera system to further improve registration accuracy.
Obviously, this is not possible in inaccessible, or dangerous environments.

Since these approaches are based on point-to-point correspondences, they require a rather high point density to achieve precise registration.
For low-cost LiDARs, this implies slow motion and long integration time.

\subsection{Point Cloud Registration Using Plane Based Correspondences}

The de-facto standard for many SLAM algorithms is the Iterative-Closest-Point (ICP) algorithm~\cite{Besl1992} that employs point-to-point correspondences using closest points, as the name suggests. 
To overcome the requirements on point-density imposed by the point-to-point corres\-pon\-den\-ces instead other correspondences are used. 
In human-made environments, planes are abundantly available and hence provide an attractive base for correspondences.

Pathak et al.~\cite{Pathak:2010} propose to reduce the complexity of the registration by using correspondences between planar patches instead of points.
They demonstrate the effectiveness of their approach even with noisy data in cluttered environments.
However, their approach is designed for data acquired in stop-scan-go fashion and not for mobile mapping applications.
As they use a region growing procedure with randomized initialization for detecting planar patches the distortions in the data introduced by pose uncertainties are likely to effect the shape of the planar patches and lead to faulty correspondences.

Förster et al. use the planarity of human-made environment successfully in~\cite{Foerstner2017}.
They register point clouds using plane-to-plane correspondences and include uncertainty measures for the detected planes and the estimated motion.
Thereby, they propose a costly exact algorithm and cheaper approximations that yield high-quality maps.
Favre et al.~\cite{favre2021} use point-to-plane correspondences after preprocessing the point clouds using plane-to-plane correspondences to register two scans with each other successfully.

Both approaches use plane-to-plane correspondences to pre-register the scans.
However, for pre-registration the classical point-to-point registration is also very effective. 
One advantage that point-to-point correspondences have over plane-to-plane correspondences is that they do not require a long stop in order to obtain enough points to detect planes in each pose.
For plane-to-plane correspondences, this is necessary to gather enough data to measure planes in each scan robustly. 
In particular this pause is needed when the field-of-view (FOV) is limited, as the detected planes are thin slices of the true planes which are difficult to find correct correspondences for. 
The resulting scan procedure is stop-scan-and-go.
In particular, for the application of a spherical robot this standstill in each pose cannot be guaranteed or even approximated, making continuous-time approaches using point-to-plane correspondences the method of choice.

Lidar odometry and mapping (LOAM)~\cite{Zhang2014} is the baseline-algorithm that provides a real-time and lowdrift solution based on two parallel registration algorithms using planes and lines.
Unfortunately, it is not open-source anymore.
LOAM livox extends the LOAM framework to the rotating prism scanner with small FoV~\cite{Lin_2020}.
Zhou, Wang, and Kaess write that it also adopts the parallel computing to achieve real-time global registration.
Parallel computing needs a powerful CPU, it may be not suitable for an embedded system which has limited computational resources~\cite{Zhou_2021}.
Thus, they extend their smoothing and mapping to the LiDAR and planes case.
In their experiments they used a VLP-16 LiDAR to collect indoor datasets.

Further recent planar SLAM approaches include~\cite{Jung_2015,Grant_2018,Geneva_2018, Zhou_2021,   wei2021groundslam}.
While Wei et al. uses only the ground plane in outdoor experiments.
Indoors, Jung et al. used in 2015 several Hokuyo laser scanners for their kinematic scanning system~\cite{Jung_2015}, while Grant used a single Velodyne HDL-32E sensor with 32 rotating laser/receiver pairs mounted on a backpack and walked through different environments~\cite{Grant_2018}.
The LIPS system~\cite{Geneva_2018} employes a so-called Closest Point Plane Representation with an Anchor Plane Factor.
Random sample consensus (RANSAC) is used to find the planes.
Their system couples an eight channel Quanergy M8 LiDAR operating at 10Hz with a Microstrain 3DM-GX3-25 IMU attached to the bottom of the LiDAR operating at 500Hz.

\bigskip

In the following, we propose this combination of point-to-point based pre-registration followed by a point-to-plane based optimization. 
