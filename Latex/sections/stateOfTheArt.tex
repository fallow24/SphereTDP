%!TEX root = ../main.tex
\section{State Of The Art}

\subsection{Spherical SLAM}

Up to the best of our knowledge, SLAM with spherical robots and laser scanners was not done before, while there exists camera-based approaches~\cite{9233654}.
However, laser-based SLAM algorithms for motions in six degrees of freedom (DoF) have been thoroughly studied.
For outdoor environments~\cite{JFR2006} provides a first baseline.
Adding a heuristic for closed-loop detection and a global relaxation Borrmann et al. yield highly precise maps of the scanned environment~\cite{RAS2007}.
Zhang et al. propose a real-time version of the SLAM algorithm in~\cite{Zhang2014}.
They achieve the performance at a lower computational load by dividing the SLAM algorithm into two different algorithms: one performing odometry at a high frequency but low fidelity and another running at a lower frequency performing fine matching and registration of the point clouds.
More recently Dröschel et al. also propose an online method using a novel combination of a hierarchical graph structure with local multi-resolution maps to overcome problems due to sparse scans~\cite{Droeschel2018}.

Since these approaches are based on point-to-point correspondences, they require a rather high point density to achieve precise registration.
For low-cost LiDARs, this implies slow motion and long integration time.
Further, none of the approaches exploit the structure inherent to human-made environments, hence missing out on possible advantages.

\subsection{Point Cloud Registration Using Plane Based Correspondences}

The de-facto standard for many SLAM algorithms is the Iterative-Closest-Point (ICP) algorithm~\cite{Besl1992} that employs point-to-point correspondences using closest points, as the name suggests. 
To overcome the requirements on point-density imposed by the point-to-point correspondences instead other correspondences are used. 
In human-made environments, planes are abundantly available and hence provide an attractive base for correspondences.
Förster et al. use this property of human-made environment successfully in~\cite{Foerstner2017}.
They register point clouds using plane-to-plane correspondences and include uncertainty measures for the detected planes and the estimated motion.
Thereby, they propose a costly exact algorithm and cheaper approximations that yield high-quality maps.
Favre et al.~\cite{favre2021} use point-to-plane correspondences after preprocessing the point clouds using plane-to-plane correspondences to register two scans with each other successfully.

Both approaches use plane-to-plane correspondences to pre-register the scans.
However, for pre-registration the classical point-to-point registration is also very effective. 
One advantage that point-to-point correspondences have over plane-to-plane correspondences is that they do not require a long stop in each pose.
For plane-to-plane correspondences, this is necessary to gather enough data to measure planes in each scan robustly.
The resulting scan procedure is stop-scan-and-go.
In particular, for the application of a spherical robot this standstill in each pose cannot be guaranteed or even approximated, making continuous-time approaches using point-to-plane correspondences the method of choice.

In the following, we propose this combination of point-to-point based pre-registration followed by a point-to-plane based optimization. 
