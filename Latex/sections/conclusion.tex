%!TEX root = ../main.tex
\section{Conclusion}

In this paper, we have proposed a registration method for man made environments that exploits the structure of those environments. 
Given that in human-made environments parallel planes and right angles are abundantly available, we employ point-to-plane correspondences to improve a pre-registered 3D point cloud. 
We have evaluated the procedure on a simulated dataset and on an experimentally acquired dataset with different motion profiles. 
In this evaluation, we have shown that the procedure improves both datasets and yields more human-made, structure-slike maps. 
In particular, the qualitative structure of the environments is reconstructed well. 
In the resulting maps, the parallel walls are clearly improved.

However, as section~\ref{sec:experimentalSetup} shows, some outliers remain due to the simplistic point-to-plane correspondence model, which is to be improved in future studies.
One idea to do so is to employ a local clustering technique onto the individual scans before registering them globally. 
Points from the scan with a similar normal get clustered together if their distance is small enough.
This way, a cluster represents a local plane while at the same time storing all the points that belong to that plane.
Point-to-plane correspondences then arise from the intersection of local clusters with global planes.

Needless to say, a lot of work remains to be done. 
%
Right now, not all steps in the procedure are autonomous, in particular the parameter tuning. 
In the future, one goal is to increase the autonomy of the system.
One approach is to introduce soft-locks for the optimization dimensions.
I.e., instead of locking some dimensions entirely from being used for optimization, they are weighted based on the dynamics of the system that encode which noise source is more likely.  
%
Curently, we are assembling the sphere of Figure~\ref{FIXME} to test it in real environments.
Unfortunately, this sphere will be too large to fit our floating setup, thus it will be tested in an office like environment.
Furthermore, we will test our rolling and scanning spheres on a rail system to achieve repeatable trajectories. 
